\section{Restructuring of the relational schema}

\noindent The obtained relational schema can be further optimized.

\medskip
Firstly, some redundant relations are found: \texttt{StopName} and \texttt{HasName}, and also \texttt{Date} and \texttt{HasExceptionOn}. In both cases, one may see that the first relations consist of just one attribute, and that the same attribute is already included in the second relations as a foreign key. It is therefore reasonable to delete \texttt{StopName} and \texttt{Date}, because the relevant data can be queried by using the \texttt{HasName} and \texttt{HasExceptionOn}.

\medskip
In general, it was noticed that the complexity of queries for interactive operations (i.e. 1-4) was quite high and it involved a considerable number of joins between relations. For this reason, the decision taken was to delete those relations deriving from relationships in the original schema, where one role had cardinality \textit{(1,1)}, for example \textit{RunsOn, PartOf, ...}. These relations have been merged into the role with cardinality \textit{(1,1)}; the inclusion, foreign key and external constraints have been modified accordingly.

\medskip
Another context in which a relation may be deleted is the one of \texttt{Stop} and \texttt{Platform}. The latter relation contains the name of the platform, plus a reference to the Stop relation. To reduce the number of joins needed to perform the various operations, it has been chosen to merge those relations together. The \texttt{Stop} relation will then gain a new \textit{nullable} attribute, \texttt{platformName}. Even though this increases the number of pages in the relation, it is also reducing the queries complexity, in particular the one of operations 3 and 4. Since those operations are interactive, more importance has been given to execution time rather than space efficiency.

\subsection{Restructured schema}

	City(\uline{name}, latitude, longitude) \\


	Company(\uline{name}, headquarterCity)

	$ \ \ $ Foreign key: Company[headquarterCity] $\subseteq$ City[name] \\


	Line(\uline{shortCode, departingCity}, color, vehicleType, description)

	$ \ \ $ Foreign Key: Line[departingCity] $\subseteq$ City[name]

	$ \ \ $ Inclusion: Line[shortCode, departingCity] $\subseteq$ Trip[lineCode, lineCity] \\


	Trip(\uline{code}, calendar, company, lineCode, lineCity)

	$ \ \ $ Inclusion: Trip[code] $\subseteq$ StopsAt[trip]

	$ \ \ $ Foreign Key: Trip[calendar] $\subseteq$ Calendar[ID]

	$ \ \ $ Foreign Key: Trip[company] $\subseteq$ Company[name]

	$ \ \ $ Foreign Key: Trip[lineCode,lineCity] $\subseteq$ Line[shortCode,departingCity] \\


	Calendar(\uline{ID}, startDate, endDate, {\small monday, tuesday, wednesday, thurdsay, friday, saturday, sunday) }

	$ \ \ $ Inclusion: Calendar[ID] $\subseteq$ Trip[calendar] \\


	Stop(\uline{latitude, longitude}, city, platformName*)

	$ \ \ $ Inclusion: Stop[latitude, longitude] $\subseteq$ StopsAt[latitude, longitude]

	$ \ \ $ Inclusion: Stop[latitude, longitude] $\subseteq$ StopHasName[latitude, longitude]

	$ \ \ $ Foreign Key: Stop[city] $\subseteq$ City[name] \\

	\newpage
	StopsAt(\uline{latitude, longitude, trip}, order, arrivalTime*, departureTime*)

	$ \ \ $ Foreign Key: StopsAt[latitude, longitude] $\subseteq$ Stop[latitude, longitude]

	$ \ \ $ Foreign Key: StopsAt[trip] $\subseteq$ Trip[code] \\


	StopHasName(\uline{latitude, longitude, language} value)

	$ \ \ $ Foreign Key: StopHasName[latitude, longitude] $\subseteq$ Stop[latitude, longitude] \\


	HasExceptionOn(\uline{date, calendar}, suppressed)

	$ \ \ $ Foreign Key: HasExceptionOn[calendar] $\subseteq$ Calendar[ID]

\subsection{External Constraints}

\begin{enumerate}
	\item For each tuple in the entity \texttt{Calendar}, the attribute \textit{endDate} must encode a date strictly greater than \textit{startDate}.

	\item For each tuple in the relationship \texttt{stopsAt}, the attribute \textit{departureTime} must encode a time greater or equal than \textit{arrivalTime}.

	\item In the relationship \texttt{stopsAt}, for the first stop of a trip (with \textit{order=1}), the attribute \textit{arrivalTime} must be omitted and \textit{departureTime} must be specified. For the last stop (with maximal \textit{order}), the attribute \textit{arrivalTime} must be specified and \textit{departureTime} must be omitted. For all the other stops, both attributes must be present.

	\item In the relationship \texttt{stopsAt}, the attribute \textit{order} must be equal to \texttt{1} for the first stop of a trip, and should be incremented by one for subsequent stops.

	\item In the relationship \texttt{stopsAt}, for every pair of tuples \textit{T1} and \textit{T2} within the same trip, if \textit{T1[order] < T2[order]}, then \textit{T1[departureTime] < T2[arrivalTime]}

	\item For a \texttt{Trip}, the exceptional dates defined through the \texttt{hasExceptionOn} relationship must be in between \textit{startDate} and \textit{endDate} defined by the corresponding \texttt{Calendar} instance. Moreover, if \textit{suppressed=True} then the corresponding day of week in the \texttt{Calendar} instance has value \textit{True}. At the opposite, if \textit{suppressed=False} then the corresponding day of week in \texttt{Calendar} has value \textit{False};

	\item In the relation \texttt{StopsAt}, the minimum number of tuples having the same value for the \textit{trip} attribute is two.
\end{enumerate}

\newpage
\subsection{Application Load}

\paragraph{(1)} Get a list of \textit{(maximum)} ten stops with name in a specific language matching the user-given string

\begin{table}[h]
	\centering
	\begin{tabular}{|c|c|c|}
		\hline
		\textbf{Concept} & \textbf{Accesses} & \textbf{Type} \\
		\hline
		StopHasName & 75 & Read \\ \hline
		Stop & 10 & Read \\ \hline
	\end{tabular}
	\caption{Access table for operation 1}\label{tbl:rest.access-1}
\end{table}

\paragraph{(2)} Get a list of \textit{(maximum)} ten stops with name in a specific language that are contained in a set of coordinates

\begin{table}[h]
	\centering
	\begin{tabular}{|c|c|c|}
		\hline
		\textbf{Concept} & \textbf{Accesses} & \textbf{Type} \\
		\hline
		Stop & 75 & Read \\ \hline
		StopHasName & 10 & Read \\ \hline
	\end{tabular}
	\caption{Access table for operation 2}\label{tbl:rest.access-2}
\end{table}

\paragraph{(3)} Find the next ten trips, with line and company information, that depart from a specific stop at a given date and time \textit{(assuming that the stop coordinates are known)}
\begin{table}[h]
	\centering
	\begin{tabular}{|c|c|c|}
		\hline
		\textbf{Concept} & \textbf{Accesses} & \textbf{Type} \\
		\hline
		StopsAt & 15 & Read \\ \hline
		Trip & 15 & Read \\ \hline
		Calendar & 15 & Read \\ \hline
		HasExceptionOn & 225 & Read \\ \hline
		Line & 10 & Read \\ \hline
	\end{tabular}
	\caption{Access table for operation 3}\label{tbl:rest.access-3}
\end{table}

\paragraph{(4)} Find the next ten trips departing from one stop and arriving at another one, at a given date and time \textit{(assuming that the stop coordinates are known)}
\begin{table}[h!]
	\centering
	\begin{tabular}{|c|c|c|}
		\hline
		\textbf{Concept} & \textbf{Accesses} & \textbf{Type} \\
		\hline
		StopsAt & 30 & Read \\ \hline
		Trip & 30 & Read \\ \hline
		Calendar & 30 & Read \\ \hline
		HasExceptionOn & 225 & Read \\ \hline
		Line & 10 & Read \\ \hline
	\end{tabular}
	\caption{Access table for operation 4}\label{tbl:rest.access-4}
\end{table}

\paragraph{(5)} Add a new company given the name and the headquarter city, \textit{and considering that all possible cities are already loaded into the database}

\begin{table}[h]
	\centering
	\begin{tabular}{|c|c|c|}
		\hline
		\textbf{Concept} & \textbf{Accesses} & \textbf{Type} \\
		\hline
		City & 1 & Read \\ \hline
		Company & 1 & Write \\ \hline
	\end{tabular}
	\caption{Access table for operation 5}\label{tbl:rest.access-5}
\end{table}

\paragraph{(6)} Add a new line given short code, color, type of vehicle and city, \textit{and considering that all possible cities are already loaded into the database}

\begin{table}[h]
	\centering
	\begin{tabular}{|c|c|c|}
		\hline
		\textbf{Concept} & \textbf{Accesses} & \textbf{Type} \\
		\hline
		City & 1 & Read \\ \hline
		Line & 1 & Write \\ \hline
	\end{tabular}
	\caption{Access table for operation 6}\label{tbl:rest.access-6}
\end{table}

\paragraph{(7)} Add a new trip given the code, the timetable, the operating company and the calendar

\begin{table}[h]
	\centering
	\begin{tabular}{|c|c|c|}
		\hline
		\textbf{Concept} & \textbf{Accesses} & \textbf{Type} \\
		\hline
		Trip & 1 & Write \\ \hline
		stopsAt & 9 & Write \\ \hline
	\end{tabular}
	\caption{Access table for operation 7}\label{tbl:rest.access-7}
\end{table}

\paragraph{(8)} Modify the schedule for a calendar knowing its code (add one exceptional date)
\begin{table}[h!]
	\centering
	\begin{tabular}{|c|c|c|}
		\hline
		\textbf{Concept} & \textbf{Accesses} & \textbf{Type} \\
		\hline
		HasException & 1 & Write \\ \hline
	\end{tabular}
	\caption{Access table for operation 8}\label{tbl:rest.access-8}
\end{table}

\section{SQL Specification}

	Three attatched files are present in the \texttt{sql} directory:
	\begin{itemize}[itemsep=0pt]
		\item \texttt{base.sql} contains the definition of the various relations
		\item \texttt{data.sql} contains some sample data
		\item \texttt{full.sql} is the dump of the complete database, left for fallback
	\end{itemize}

	They may be imported into a local instance with the following commands:
	\begin{center}
		\begin{lstlisting}
			psql
			postgres=# \i base.sql
			publictransport_17573=# \i data.sql
		\end{lstlisting}
	\end{center}

	As it may be seen, tuple and external constraints have been implemented either as CHECK operations or as triggers with stored procedures. Note that external constraints 5-6 were considered too difficult and as such they have not been implemented. Constraint 4 was implemented by relying on postgres' \texttt{Serial} data type, which represents an increasing integer that start from 1. The domain \texttt{VehicleType} for a \textit{Line} has been created as a custom type.

\subsection{Queries related to the operations}

	\paragraph{(1)} Get a list of \textit{(maximum)} ten stops with name in a specific language matching the user-given string. \textit{Parameters: stop name, language}
	\begin{center}
		\begin{lstlisting}
	SELECT S.city, SN.value AS name, S.latitude, S.longitude
	FROM Stop S, StopHasName SN
	WHERE S.latitude=SN.latitude AND S.longitude=SN.longitude
	AND SN.value ILIKE _name_ AND SN.language=_language_;
		\end{lstlisting}
	\end{center}

	\paragraph{(2)} Get a list of \textit{(maximum)} ten stops with name in a specific language that are contained in a set of coordinates. \textit{Parameters: latitude, longitude, radius, language}
	\begin{center}
	\begin{lstlisting}
	SELECT S.city, SN.value AS name, S.latitude, S.longitude
	FROM Stop S, StopHasName SN
	WHERE SN.latitude=S.latitude AND SN.longitude=S.longitude
	AND acos( cos(radians(_latitude_)) * cos(radians(S.latitude)) *
	cos(radians(S.longitude) - radians(_longitude_)) + sin(radians(_latitude_)) *
	sin(radians(S.latitude)) ) * 6371393 <= _radius_
	AND S.platformName IS NULL AND SN.language=_language_;
	\end{lstlisting}
\end{center}

	\newpage
	\paragraph{(3)} Find the next ten trips, with line and company information, that depart from a specific stop at a given date and time. \textit{Parameters: stop latitude, stop longitude, date and time}
	\begin{center}
		\begin{lstlisting}
	SELECT T.company, L.shortCode AS Line, L.vehicleType, T.code as "Trip Code",
	  SA.departureTime AS "Departure Time"
	FROM trip T, calendar C, stopsAt SA, line L
	WHERE SA.latitude=_stopLat_ AND SA.longitude=_stopLon
	AND SA.departureTime>=_time_ AND T.calendar=C.ID
	AND T.code=SA.trip AND T.lineCode=L.shortCode
	AND (
	((C._dayOfWeek_=True) AND (_date_ NOT IN (SELECT date FROM HasExceptionOn E
		WHERE suppressed=True AND E.calendar=C.ID)))
	OR (_date_ IN (SELECT date FROM HasExceptionOn E
		WHERE suppressed=False AND E.calendar=C.ID))
	)
	ORDER BY SA.arrivalTime
	LIMIT 10;
		\end{lstlisting}
	\end{center}

	\paragraph{(4)} Find the next ten trips departing from one stop and arriving at another one, at a given date and time. \textit{Parameters: latitude and longitude for the two stops, date and time}
	\begin{center}
		\begin{lstlisting}
	SELECT T.company, L.shortCode AS Line, L.vehicleType, T.code as "Trip Code",
	  S1.departureTime AS "Departure Time", S2.arrivalTime AS "Arrival Time",
	  (S2.arrivalTime-S1.departureTime) AS duration
	FROM trip T, calendar C, stopsAt S1, stopsAt S2, line L
	WHERE S1.latitude=_s1lat_ AND S1.longitude=_s1lat_ AND S2.latitude=_s2lat_
	AND S2.longitude=_s2lon_ AND S1.departureTime>=_time_
	AND S1.trip=S2.trip AND T.calendar=C.ID AND T.code=S1.trip
	AND T.lineCode=L.shortCode
	AND (
	((C._dayOfWeek_=True) AND (_date_ NOT IN (SELECT date FROM HasExceptionOn E
		WHERE suppressed=True AND E.calendar=C.ID)))
	OR (_date_ IN (SELECT date FROM HasExceptionOn E
		WHERE suppressed=False AND E.calendar=C.ID))
	)
	ORDER BY S1.arrivalTime
	LIMIT 10;
		\end{lstlisting}
	\end{center}

	\paragraph{(5)} Add a new company given the name and the headquarter city. \textit{Parameters: company name, company city}
	\begin{center}
		\begin{lstlisting}
		INSERT INTO Company(name,headquarterCity) VALUES(...);
		\end{lstlisting}
	\end{center}

	\paragraph{(6)} Add a new line given short code, color, type of vehicle and city. \textit{Parameters: line short code, departing city, color, vehicle type}
	\begin{center}
		\begin{lstlisting}
	INSERT INTO Line(shortCode, color, vehicleType, cityName) VALUES (...);
		\end{lstlisting}
	\end{center}

	\paragraph{(7)} Add a new trip given the code, the timetable, the operating company and the calendar. \textit{Parameters: trip code, calendar ID, line code and departing city, company name. List of stops with the timetable (arrival and departure time)}
	\begin{center}
		\begin{lstlisting}
	INSERT INTO Trip(code,calendar,lineCode,lineCity,company) VALUES (...);
	INSERT INTO StopsAt(latitude,longitude,arrivalTime,departureTime) VALUES (...);
		\end{lstlisting}
	\end{center}

	\paragraph{(8)} Modify the schedule for a calendar knowing its code (add one exceptional date). \textit{Parameters: date, calendar ID, suppression state}
	\begin{center}
		\begin{lstlisting}
	INSERT INTO HasExceptionOn(date, calendar, suppressed) VALUES (...);
		\end{lstlisting}
	\end{center}

\section{Java Applcation}

	The Java code is available in the \texttt{app/code.zip }file. A ready-to-be-run file is found in the \texttt{app} directory. It may be executed in a terminal with:
	\begin{center}
		\begin{lstlisting}
		java -jar app/publictransport_17573.jar
		\end{lstlisting}
	\end{center}

	Note that Java 15 is required. It may be needed to specify the path to postgresql driver with \texttt{-classpath} argument. The connection parameters may be altered by modifying the \texttt{database.properties} file.

	The application allows to execute the queries related to the table of operations. After the initialization, the user is presented with a menu, from which all operations can be run:

	\begin{center}
		\centering
		\begin{lstlisting}
		****** Introduction to Databases project ******

		Connecting... successfully connected to the database


		------------ MENU ------------
		1) Stops by name
		2) Stops by coordinates
		3) Trips per Stop
		4) Trips between two stops
		5) Add new company
		6) Add new line
		7) Add new trip
		8) Modify calendar schedule
		9) Exit
		------------------------------
		Please choose an item:
		\end{lstlisting}
	\end{center}
